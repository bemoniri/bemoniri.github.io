\chapter*{}
\begin{center}
\vspace{0.9cm}
\large
\textbf{چکیده}
\end{center}
در  این پروژه، ابتدا به تعریف مسئله‌ی جداسازی کور منابع می پردازیم. بعد از مرور قضایای جدایی‌پذیری در مخلوط های خطی، چالش‌های موجود در جداسازی مخلوط‌های غیر خطی را مطرح کرده و نشان می‌دهیم در کلّی‌ترین حالت ممکن، جداسازی کور مخلوط‌های ترکیب‌های غیرخطی با روش‌های مبتنی بر استقلال متغیّر‌های تصادفی ممکن نیست و برای این جداسازی، نیاز به وجود شرطی قوی‌تر از استقلال متغیر‌های تصادفی بر روی منابع داریم. با توجه به نتایج تجربی موجود در ادبیات این حوزه، ایده‌ی استفاده از وابستگی‌های زمانی‌ سیگنال‌ها را مطرح کرده و تلاش می‌کنیم با بهره بردن از آن، قضیه‌ای عمومی برای جداسازی مخلوط‌های غیرخطی ارائه کنیم و  به بررسی نتایج جدید مبتنی بر این ایده می‌پردازیم. در خلال این کار، به بیان حالاتی خاص از توابع غیرخطی و ناهموار که حتی با وجود فرض استقلال فرآیند‌های تصادفی ورودی (یا منابع) لزوماً قابل جداسازی نیستند می‌پردازیم.  الگوریتمی بر مبنای کمینه‌سازی اطلاعات متقابل برای جداسازی مخلوط‌های غیرخطی ارائه می‌کنیم و عملکرد آن را در جداسازی مخلوطی خاص مورد بررسی قرار می‌دهیم. بعد از مروری بر 
\lr{HSIC}
به عنوان جایگزینی احتمالی برای اطلاعات متقابل در الگوریتم‌های جداسازی کور منابع، در آخر  به بیان راه‌‌های پیش رو برای پژوهش در درس پروژه کارشناسی ۲ خواهیم پرداخت.




\vspace{0.3cm}
\noindent\textbf{کلمات کلیدی:}\noindent
\textbf
{جداسازی کور منابع، فرآیند‌های تصادفی، اطلاعات متقابل، استقلال، تحلیل مؤلفه‌های مستقل، روش‌های کرنل}.
