\section
[دو کاربرد نامساوی‌های مبتنی بر مارتینگل]
{دو کاربرد از نامساوی‌های مبتنی بر مارتینگل}

\subsection{عدد رنگی گراف‌های تصادفی}
یک گراف تصادفی
$G_{(n, p)}$
را یک گراف تصادفی با $n$ رأس تعریف می‌کنیم که بین هر دو رأس آن، با احتمال 
$p$
یک یال وجود دارد. 

برای یک گراف داده‌‌شده‌ی 
$G$،
عدد رنگی گراف برابر است با کمینه‌ی ‌‌تعداد رنگ‌های لازم برای رنگ‌‌آمیزی رأس‌های $G$، به این ترتیب که هیچ دو رأس مجاور گراف، هم‌رنگ نباشند.  این عدد را با 
$\chi(G)$
نمایش می‌دهیم. می‌دانیم که پیدا کردن عدد رنگی گراف، یک مسئله‌ی 
\lr{NP-Hard}
است. در این تمرین قصد داریم نشان دهیم که 
$\chi(G_{(n, p)})$
حول میانگین خود، تجمع دارد.  برای این کار، مارتینگل 
\lr{Doob}
زیر را در نظر بگیرید:
$$Z_i \triangleq \mathbb{E}[\chi(G_{(n, p)})|G_1, \dots, G_i]$$
که در آن $G_i$  زیر‌گراف با رأس‌های
$\{1, \dots, i\}$
از گراف اصلی است. توجه کنید که 
 $Z_0 = \mathbb{E}[\chi(G)]$.
 
 \begin{enumerate}
 \item
 نشان دهید که 
$|Z_{i+1} - Z_i|<1$
است.
\item 
به کمک نامساوی 
\lr{Azuma-Hoeffding}،
نشان دهید که عدد رنگی این گراف با احتمال بالا در فاصله‌ی
$O(\sqrt{n\log(n)})$
از امید ریاضی‌اش قرار دارد.
 \end{enumerate}

\subsection{\lr{Balls and Beans!}}
 فرض ‌کنید  $n$ توپ را در $n$  سبد پرتاب می‌کنیم و قصد داریم تعداد سبد‌های خالی را مطالعه کنیم. متغیر $X_i$ را شماره‌ی سبدی که توپ $i$ در آن افتاده است بگیرید. همچنین $Y$ را برابر تعداد سبد‌های خالی تعریف کنید. 
 
 \begin{enumerate}
 \item
با تعریف مارتینگل 
\lr{Doob}
 مناسب، کرانی برای 
$\mathbb{P}(Y - \mathbb{E}[Y] \geq \epsilon n)$
 ارائه دهید.
 \item 
امید ریاضی $Y$  را بیابید.
 \end{enumerate}
 